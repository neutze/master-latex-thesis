\subsection{Developers} \label{subsection:foundation-piracy-developers}
Especially for software developers piracy is a problem.
The most apparent issue is regarding lost revenue and clear at first glance.
\newline
When people are downloading an application for free and do not pay for it, there is no revenue generated and the developer does not earn any money.
\newline
At second glance, the facts are more complex.
Income is not only lost when the user is not paying on purchase, the pirate can also modify the application to influence the follow up revenue.
An example for the loss of future earnings is the unlocking of inapp purchases.
Inapp purchases are a popular source of income for so called freemium games or lite versions of apps.
The application itself is for free but has, for example, an ingame currency (freemium) respectively limited features (lite version).
The revenue is generated by offering additional ingame currency (freemium) or the unlocking of all features (lite version) in exchange for money.
When the inapp purchases are unlocked, the process executed with success but no payment is transfered.
Thus no earnings are generated for the developer.
\newline
An different example for the redirection of revenue is the modification of the Ad Unit ID \cite{googleAdmob}.
The Ad Unit ID is responsible to assign earnigns generated by an mobile advertisment to the developer.
When an application is pirated, this code can be replaced by the pirate's one. Future revenues generated by advertisments in the application will not be assigned to the developer but the pirate.
\newline
\newline
Additional problems arise when the app is taken from the environment of an official app store and moved to a blackmarket store or website.
The first is the loss of control over the application.
The developer can no longer provide support or updates for the application in case of malfunction.
The second one is that users connect maliscious behaviour to the developer and not to the pirated version.
Both these issues cause mistrust in the developer and might influence future revenues which are not even connected to this applciation.
The third problem is unpredictable traffic.
When distributing an application over Google Play the developer can monitor growth and adjust eventual servers accordingly.
But when the application is distributed over unknown sources the developer cannot evaluate, the traffic can increase dramatically and cause a bad user experience for legitimate users as well \cite{lierschDeveloperThreats}.
\newline
\newline
Developers have to live of their applications.
When they do not earn money, either because the revenue stream is redirected or because their \gls{ip} is stolen and commercialized by someone else, they cannot continue with developing and their skills are lost.



%START TEXT INPUT

%on the first view lost revenues when the application is not bought anymore but downloaded from a blackmarket or website
%on the second view the application can not only be made available without purchasing it, but it can also be modified and misused further
%- the attacker can replace the developer's google ad id with his own and this way the ad revenues go to his pockets and are missing in the developers one
%- disable the verification for in app sales, this way the user can buy items from the e.g. inapp store without being charged and the developer not earnign anything, or even more direct: simply activate the pro feature for application in the code

%overall when the app is taken from the environment of the app store it is sold  and moved to blackmarket/website to download, the developer has no more control over it, he cannot support it with fixes and updates
%when this happens or maliscious behaviour was included by the pirate, the bad reputation is on the developer
%also the generated traffic cannot be predicted anymore, this traffic does not make money but costs us, double fail

%and if the code is stolen in general, it can be reverse engineered,  the awesome algorithms included are lost and can be used in any way the pirate likes
%\cite{lierschDeveloperThreats}
