\subsection{Android} \label{subsection:foundation-piracy-android}
Piracy is widespread on the Android platform. Especially in countries like China piracy is as high as 90\% due to restricted access to Google Play \cite{piracyRate}.
Sources for pirated applications can be easily found on the internet.
Simple searches containing \textit{free apk} and the applications name return plenty of results on Google Search.
The links direct to blackmarket applications, as Blackmart \cite{blackmartStore}, and websites for cracked \gls{apk}, as crackApk \cite{crackApk}.
These providers claim to be user friendly because they offer older versions of applications or do not charge money for complete version of applications.
Their catalog includes premium apps which are not free in the Play Store and include license verification mechanisms.
This is only possible when the license mechanism is cracked \cite{apksfree}.
They practice professional stealing and are dangerous for users (see section~\ref{subsection:foundation-piracy-users}).
When downloading an \gls{apk} it is not possible to guarantee the integrity of the program.
\newline
An example for the dimensions piracy can reach for a single application is \textit{Today Calendar Pro}.
The developer stated in a Google+ post that the piracy rate of the application is as high as 85\% at the given day.
This results in only 15\% being legitimately purchased and installed. \cite{xdaPiracy}\cite{developersPiracy}
\newline
Since license mechanisms are no obstacle for pirates, some developers do not implement any copy protection at all since it is cracked within days \cite{recodeMonument}.
Especially Android applications are at high risk for piracy due bytecode in general is an easy target to reverse engineer.
