\subsection{Android Application Package (APK)} \label{subsection:foundation-android-package}
Android applications are distributed and installed using the \gls{apk} file format.
They can either be obtained from an application store, like Google Play, or downloaded and installed, manually or by using \gls{adb}, from any other source.
The \gls{apk} format is based on the ZIP file archive format and contains the code and resources of the application.
\newline
The build process of \gls{apk} contains several steps which are visualized in figure~\ref{fig:apk}.
\newline
\begin{figure}[h]
    \centering
    \includegraphics[width=0.8\textwidth]{data/apk.png}
    \caption{\gls{apk} build process \cite{andevconDalvikART}}
    \label{fig:apk}
\end{figure}
Since Android applications are usually written in Java, the start is similar to the Java program build process.
Android applications are usually written and Java.
They have the same build process as standard Java applications.
Upon compilation, the source code is compiled to \gls{classg} files by the Java Compiler javac.
Each Java class is stored as bytecode in the corresponding \gls{classg} file.
Java bytecode can be obfuscated, which is topic of section~\ref{subsection:counter-improve-obfuscation}.
When all Java classes are compiled to \gls{classg} files, they are packed into a \gls{jar} file.
\newline
Since Android is using a different \gls{vm} for executing the code, the Java bytecode has to be converted to Dalvik bytecode.
The Android \gls{sdk} provides \textit{dx}, the tool used to convert \gls{classg} files to a single \textit{classes.dex} file.
The \gls{vm} and the \gls{dex} file format will be described in the following.
Similar to the Java bytecode, obfuscation can be applied.
\newline
The \gls{apk} itself consits of three parts.
\begin{itemize}
\item \textit{classes.dex}, containing the bytecode
\item resource files (\textit{res/*.*}), containing static content like images, the strings.xml and the layout.xml files
\item resources.arsc and AndroidManifest.xml, containing compiled resources respectively essential information as required permissions
\end{itemize}
The \textit{apkBuilder} combines these files into one archive file. 
\newline
Before releasing the application, it has to be signed and zipaligned.
The \textit{jarsigner} is used to sign the application with the private key of the developer.
It enables Android to identify the developer and support future updates for the application.
Afterwards \textit{zipalign} is used to mark uncompressed data. \cite{androidPublishSign} \cite{andevconDalvikART}
\newline
\newline
The structure of a final \gls{apk} file has at least the following content seen in figure~\ref{fig:apkfolder}.
\begin{figure}[h]
    \centering
    \includegraphics[width=0.5\textwidth]{data/apkfolder.png}
    \caption{\gls{apk} folder structure}
    \label{fig:apkfolder}
\end{figure}
The AndroidManfiest.xml and the \textit{classes.dex}, which have been covered already.
The META-INF folder, which is inherited from Java and used to store package and extension configuration data, e.g. the signature \cite{metaJava}.
While the static resources, like drawables and layouts, are in the res folder, the resources.arsc contains the compiled resources.
In case the application implements native code, it is stored in the libs folder, split by the different processor types, like armeabi-v7a for ARM or x86 for Intel processors. \cite{kovachevaMaster} \cite{ehringerDalvik}
