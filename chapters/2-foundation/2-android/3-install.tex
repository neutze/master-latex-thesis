\subsection{Installing an APK} \label{subsection:android-install}
Now that the format of the \gls{apk} and \gls{dex} file format are explained, the application can be installed.
Before running an application, two steps are applied on the \gls{apk}.
The first step is primary about the verification of the application while the second step is the bytecode optimization and, in case of \gls{art}, the code compilation.
Before installing an application it is checked for a legitimate signature as well as correct classes.dex structure (see figure~\ref{fig:install}).
In case this cannot be verified, the installation will be rejected by the OS.
The second step is the optimization.
In case the device is still running the \gls{dvm}, the \gls{odex} verison of the classes.dex is generated.
This is possible because \gls{dex} files can be optimized in order to achieve the best performance for a given device architecture.
This is due to the the high diversity of Android running hardware and their different processors.
In the process the classes.dex file is taken from the archive and put as \gls{odex} file into the Dalvik cache.
This is done once and from then on the execution is done by using the \gls{odex} file.
This preprocessed version of the application has an improved startup time. \cite{kovachevaMaster}
The current versions of Android run on the \gls{art}.
For this runtime the second step is more complex since the bytecode has to be compiled an additional time.
This will be explained closer in section~\ref{subsection:android-art}.
\newline
\begin{figure}[h]
    \centering
    \includegraphics[width=0.8\textwidth]{data/install.png}
    \caption{Installing an \gls{apk} on a device \cite{googleIOArt}}
    \label{fig:install}
\end{figure}

When the application is run later, Android is creating an sandboxed environment for each application.
This is achieved by Android assigning each process a seperate user ID on install to ensure that each application is isolated from the others and has no access on resources except its own. \cite{developerFundamentals}
