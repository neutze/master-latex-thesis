\subsection{Android Runtime} \label{subsection:android-art}
In Android version 4.4 \textit{Kitkat} Google introduced \gls{art}, designed to replace the \gls{dvm}.
It was optional first and only available as a preview through the developer options.
\newline
For backwards compatibility, \gls{art} still works with bytecode in the \gls{dex} files format \cite{androidArt}.
With the release of version 5.0 \textit{Lollipop} \gls{art} it became the runtime of choice since \gls{dvm} had some major flaws.
Throughout the Android 6.0 \textit{Marshmallow} previews it was constantly evolving and sometimes breaking compatibility with older versions.
In addition, almost no documentation was available.
\newline
\gls{art} is designed to address the shortcomings of the \gls{dvm}.
\begin{itemize}
\item expensive \gls{vm} maintenance - wasteful \gls{jit} and too many CPU cycles by background threads
\item frequent hangs and jitters caused by the \gls{gc}
\end{itemize}
This can be directly translated to increased battery usage and slower performance.
The 32 bit support of the \gls{dvm} look like a disadvantage because iOS already supports 64 bit, but it is not.
\newline
The improvements in \gls{art} make the maintenance less expensive, like replacing \gls{jit} with \gls{aot} and reducing overhead cycles.
The \gls{gc} is also non-blocking now and and can run parallel in fore- and background.
\newline
The main idea of \gls{art} and \gls{aot} is to compile the application to one of two types, either native code or \gls{llvm} code.
Each of the types has its purpose and advantage.
The native code runs directly on the hardware it was created for and offers improved execution performance while the \gls{llvm} code offers portability.
In practice the preference is to compile to native since adding \gls{llvm} bitcode adds another layer of complexity to the architecture.
\newline
Different from \gls{dvm}, \gls{art} uses not one but two file formats.
Similar to the zygote of \gls{dvm}, \gls{art} offers an image of pre-initialized classes and related objects at run time, the boot.art file.
It is poorly documented and still changing a lot.
The boot.art file is mapped in memory before the linked .oat file
It is mapped to the memory upon zygote startup to provide improved application starting time .
In addition to the boot.art file, there are two different .oat files.
The boot.oat contains around fourteen of the most used Android framework \gls{jar}s.
The other .oat files are the former \gls{odex} files.
They are still located in the Dalvik cache, but they are now \gls{elf} files with the odex file embedded.
Instead of \textit{dexopt}, \textit{dex2oat} is used to create these files. \cite{andevconDalvikART} \cite{developersConfigureArt} \cite{androidArt} \cite{intelArt}
\newline
In general, there is still room for improvement since not all code is guaranteed to be compiled.
Since the base code is still dex and thus the \gls{vm} is still 32 bit, \gls{art} is not fully 64 bit.
The generated code is also not always as efficient as from a native compiler. The \gls{art} compilation process is likely to be improved as \gls{art} evolves.
