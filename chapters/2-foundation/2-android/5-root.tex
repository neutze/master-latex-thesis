\subsection{Root on Android} \label{subsection:android-root}
%START TEXT INPUT
This is my real text! Rest might be copied or not be checked!
%START TEXT INPUT

%
what is rooting?
getting root/rooting process of modifying the operation system that shipped with your device to grant you complete control over it
overcome limitations by carriers/manufacturers, extend system functionality, upgrade to custom flavor
root comes from Linux OS world, most privileges user on the system is called root
rooting fairly simple, many videos and tutorials, sometimes oneclick tools
not aproved by manufacturers or carriers, can not prevent usually exploits vulnerability in operating system code or device drivers, allows the "hacker" to upload a special program called su to the phone
su provides root access to programs that request it
usually superuser permissions bundled with root
approve/deny requests from applications who want root
replaces convetional password with approve/deny, not secure but much more convenient
rooting the phone modifies the software thus can brick the phone, meaning the phone is nonfunctional since the software is broken

benefits
access all files on the phone, even those which the normal user has no permissions for, modify add delete

examples
modify system variables, e.g. to utilize the notification led on motorola devices which is usually disabled
install custom roms
\cite{androidpoliceRoot}
%

%
android's content protection are invalid when rooted
any application's data directory (or code) can be read
DRm can be bypassed
coupled with dex decompilkation big problem, app can be decompiled, modded and repackaged
\cite{levinAndevcon}
%

a list of root vulnerabilities can be found on \url{http://androidvulnerabilities.org/all}
