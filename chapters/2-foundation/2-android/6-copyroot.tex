\subsection{Copy Protection and Root} \label{subsection:android-root}
%START TEXT INPUT
unauthorized usage of an app through copy protection
apk was installed in a location on the phone /data/app user could not access
useless if single user can get apk and redistribute, gained by root
successful as long as root was not easy for everyone -see- samsung rooting odin QUELLE, did not have too big impact as today, oneclick root kits QUELLE, standard for nexus

on install app's classes.dex is copies an optimized version (odex) to dalvik cache /data/dalvik-cache/, for faster startup (contains system apps and frameworks as well)
odex erklären, byteswapping, structure realigning and memory-mapping
when app is started optimized code from dalvik cache will be executed instead of apk
old system was to put APK in folder user cannot access, unless you root
weak, almost non-existant, anyone who can copy it can distribute it

It replaces the old system copy protection system, wherein your APKs would be put in a folder that you can't access. Unless you root. Oh, and anyone who can copy that APK off can then give it to someone else to put on their device, too. It was so weak, it was almost non-existant.
kann mit root umgangen werden

Im original vom Markt direkt rutnergeladen und dann wird sie an den ort geschoben und kann nicht mehr zugegriffen werden -see- rechte etc, QUELLE



getting root/rooting process of modifying the operation system that shipped with your device to grant you complete control over it
overcome limitations by carriers/manufacturers, extend system functionality, upgrade to custom flavor
root comes from Linux OS world, most privileges user on the system is called root
rooting fairly simple, many videos and tutorials, sometimes oneclick tools
not aproved by manufacturers or carriers, can not prevent usually exploits vulnerability in operating system code or device drivers,a list of root vulnerabilities can be found on \url{http://androidvulnerabilities.org/all}, allows the "hacker" to upload a special program called su to the phone,
su provides root access to programs that request it
usually superuser permissions bundled with root
approve/deny requests from applications who want root
replaces convetional password with approve/deny, not secure but much more convenient
rooting the phone modifies the software thus can brick the phone, meaning the phone is nonfunctional since the software is broken

rooting has benefits
access all files on the phone, even those which the normal user has no permissions for, modify add delete

examples
modify system variables, e.g. to utilize the notification led on motorola devices which is usually disabled
install custom roms

but there are downsides as well
android's content protection are invalid when rooted
DRm can be bypassed
coupled with dex decompilkation big problem, app can be decompiled, modded and repackaged\cite{levinAndevcon}

%START TEXT INPUT
%
%unauthorized usage of an app through copy protection
%apk was installed in a location on the phone /data/app user could not access
%useless if single user can get apk and redistribute, gained by root
%successful as long as root was not easy for everyone -see- samsung rooting odin QUELLE, did not have too big impact as today, oneclick root kits QUELLE, standard for nexus

%on install app's classes.dex is copies an optimized version (odex) to dalvik cache /data/dalvik-cache/, for faster startup (contains system apps and frameworks as well)
%odex erklären, byteswapping, structure realigning and memory-mapping
%when app is started optimized code from dalvik cache will be executed instead of apk
%\cite{munteanLicense}
%
%
%what is rooting?
%getting root/rooting process of modifying the operation system that shipped with your device to grant you complete control over it
%overcome limitations by carriers/manufacturers, extend system functionality, upgrade to custom flavor
%root comes from Linux OS world, most privileges user on the system is called root
%rooting fairly simple, many videos and tutorials, sometimes oneclick tools
%not aproved by manufacturers or carriers, can not prevent usually exploits vulnerability in operating system code or device drivers, allows the "hacker" to upload a special program called su to the phone
%su provides root access to programs that request it
%usually superuser permissions bundled with root
%approve/deny requests from applications who want root
%replaces convetional password with approve/deny, not secure but much more convenient
%rooting the phone modifies the software thus can brick the phone, meaning the phone is nonfunctional since the software is broken

%benefits
%access all files on the phone, even those which the normal user has no permissions for, modify add delete

%examples
%modify system variables, e.g. to utilize the notification led on motorola devices which is usually disabled
%install custom roms
%\cite{androidpoliceRoot}
%

%
%android's content protection are invalid when rooted
%any application's data directory (or code) can be read
%DRm can be bypassed
%coupled with dex decompilkation big problem, app can be decompiled, modded and repackaged
%\cite{levinAndevcon}
%

%a list of root vulnerabilities can be found on \url{http://androidvulnerabilities.org/all}

%old system was to put APK in folder user cannot access, unless you root
%weak, almost non-existant, anyone who can copy it can distribute it


%It replaces the old system copy protection system, wherein your APKs would be put in a folder that you can't access. Unless you root. Oh, and anyone who can copy that APK off can then give it to someone else to put on their device, too. It was so weak, it was almost non-existant.\newline
%kann mit root umgangen werden


%Im original vom Markt direkt rutnergeladen und dann wird sie an den ort geschoben und kann nicht mehr zugegriffen werden -see- rechte etc, QUELLE\newline
