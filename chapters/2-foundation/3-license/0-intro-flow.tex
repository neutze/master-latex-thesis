\begin{itemize}
    \item test
\end{itemize}
Since the original approach of subsection~\ref{subsection:android-copyroot} was voided, another method had to be introduced.
This topic is not only relevant to Google as the main of contributor to Android and provider of its biggest store.
Since Android allows to install apps from unknown sources and not only the Google Play, other stores were created to get a piece of Google's Android pie.
Some of the most widespread stores are from Amazon and Samsung.
Amazon does not only have the Amazon Store but is also trying to create an own ecosystem by selling the Firetablets with an Amazon tailored flavor of Android at a low price tag.
Another approach is Samsung pursuiting. In addition to a store they are also offering different services as well to bind to their ecosystem.
There are different Chinese stores as well but they are out of scope since their relevance is bound to the eastern markets.
\newline
All these stores have to fight the copy protection problem in order make their store attractive and bind developers with low piracy rates.
Since the initiative is coming from the stores, the copy protection methods are included into application itself. \cite{munteanLicense}




%Es muss generell immer abgewogen werden zwischen Reichweite und Sicherheit. Von Output den Lucky Patcher gibt, sind die auto patching modes für Google, Amazon und Samsung, die großen Player. Ein Developer muss seine App dort anbieten um Aufmerksamkeit zu bekommen. Deswegen sind diese Stores auch so gut "maintained" von Lucky Patcher.
%Im Falle, dass ein Developer "Sicherheit" vorzieht und seine App in einem alternativen Store anbietet, gibt es zwei Scenarien. Entweder entwickelt jemand einen Custom Patch (dex oder native Angriff) wenn ein "allgemeines Interesse" besteht oder die App ist uninteressant und erhält keine Aufmerksamkeit, weder von LP noch Kunden.
