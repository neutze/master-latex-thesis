Since the original copy protection of subsection~\ref{subsection:android-copyroot} can be circumvented, a new and secure protection for Android applications is needed.
This is not only relevant to Google as the main of contributor to Android and provider of its biggest store.
Since Android philosophy is to allow the installation of apps from any source and not only the Google Play, other stores were created to get a piece of Google's Android business.
Some of the most widespread stores are from Amazon and Samsung.
Amazon does not only have the Amazon Store but is also trying to create their own ecosystem by selling the \textit{Fire tablets}.
They use a flavor of Android tailored to fit Amazon's needs and come at a low price.
Samsung pursues a different approach.
In addition to a store, they are also offering different services to bind to their ecosystem.
There are different Chinese stores as well.
They are out of scope here since they have no relevance in the western markets and most of them do not implement any kind of copy protection.
\newline
All these stores have to fight the copy protection problem in order make their store attractive and attract developers with low piracy rates.
