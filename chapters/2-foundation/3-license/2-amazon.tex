\subsection{Amazon DRM} \label{section:license-amazon}
Amazon started its own application store in October 2010 \cite{amazonBeta} as an alternative to Google Play.
The Amazon appstore opened to the public on the 03/22/2011 \cite{amazonRelease}.
It can be used on all Android devices and \textit{Fire} tablets.
The store comes with its own \gls{drm} since the Google LVL only works with the Google Play Store.
The \gls{drm} is called \textit{Kiwi} as it can be seen in the reverse engineered code in figure~\ref{fig:amazonFolder}.
\newline
\begin{figure}[h]
    \centering
    \includegraphics[width=0.3\textwidth]{data/amazonFolder.png}
    \caption{Amazon DRM structure of a decompiled application}
    \label{fig:amazonFolder}
\end{figure}
\newline
The prerequisites for using \textit{Kiwi} are similar to the ones of the \gls{lvl}, since the developer requires a developer account on the Amazon Developer Service platform.
According to the description, library is aimed to \grqq{Protect your application from unauthorized use. Without DRM, your app can be used without restrictions by any user.}\grqq \cite{amazonDeveloper}.
\newline
Amazon has a different approach for implementing the license verification library.
Instead of providing the developer the source code, Amazon injects the mechanism automatically into the application when it is uploaded.
The developer can chose in the developer console whether this should be done or not (see figure~\ref{fig:amazon}).
In order to implement the library, the \gls{apk} is decompiled on the server side, the library is added and the application is compiled again.
This requires the package to be signed with a new signature as described in subsection~\ref{subsection:foundation-android-package}.
Instead of using the developer's own key, Amazon uses a developer specific key.
The key itself is available on the developer platform as seen in figure~\ref{fig:amazon}. \cite{amazonDeveloper}
\newline
\begin{figure}[h]
    \centering
    \includegraphics[width=1\textwidth]{data/amazon.png}
    \caption{Developer preferences in the Amazon developer console \cite{amazonDeveloper}}
    \label{fig:amazon}
\end{figure}
\newline
The implementation can be analysed with reverse engineering.
The license verification library is wrapped around the original launcher activity of the application.
Its logic is not interweaved with application logic.
The original \textit{onCreate()} method, which is called when the application is started, is renamed to \textit{onCreateMainActivity()} and a new \textit{onCreate()} is injected.
The new method can be seen in code snippet~\ref{codeSnippet:amazonCreate}.
When the application is launched, not only the application is initiated as before, but also the \textit{Kiwi} \gls{drm} functionality is started by calling \textit{Kiwi.onCreate((Activity) this, true)}.
\newline
\lstinputlisting[
  float=h,
  basicstyle=\footnotesize,
  breakatwhitespace=false,
  breaklines=true,
  captionpos=b,
  frame=single,
  numbers=left,
  language=Java,
  linerange={77-80},
  firstnumber=77,
  caption={Amazon DRM injection in the \textit{onCreate()}},
  label={codeSnippet:amazonCreate}
]{data/amazon.java}
The license verification requires the Amazon's Appstore application to be installed on the device.
Similar to the Google Play server, the store application has the knowledge of whether the user has purchased the application and thus is allowed to use it or not.
As soon as the user logs in once into the Appstore application, the information is downloaded to the device.
In case Amazon's store is installed on the device, but the user is not signed in, the application prompts the user to sign in.
Since signing in requires an online connection to the Amazon server, \textit{Kiwi} is depending on an internet connection as well.
It is different when the wrong user is signed in or the store is not even installed.
In this case, the application shows a warning that the app is not owned by the current user, respectively that the Amazon Appstore is required and cannot be found.
