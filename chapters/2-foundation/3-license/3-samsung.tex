\subsection{Samsung DRM (Zirconia)} \label{section:license-samsung}
Another major player in the smartphone business is Samsung \cite{comscoreMarket}.
With \textit{GalaxyApps}, renamed from \textit{SamsungApps} in July 2015, they offer an application store to their Android devices.
Application distributed in that store can be protected using \textit{Zirconia} \cite{samsungZirconia}.
\newline
The way the library works is similar to the \gls{lvl}.
The library queries the Samsung server to verify the license of the user in order to prevent unauthorized usage of the application.
The library can be downloaded from Samsung in an archive file  \cite{samsungZirconia}.
It contains the compiled Zirconia library as a \gls{jar} and additional native libraries.
The integration requires both file types to be added to the application.
\newline
The implementation in the application code is done the same way as in the \gls{lvl}.
The developer is free where to implement the three code additions needed.
\newline
First of all the required permissions have to added to the \textit{AndroidManifest.xml}.
Zirconia needs access to the internet and to the phone state (see code snippet~\ref{codeSnippet:zirconiaPermission}).
\newline
\lstinputlisting[
  float=h,
  basicstyle=\footnotesize,
  breakatwhitespace=false,
  breaklines=true,
  captionpos=b,
  frame=single,
  numbers=left,
  language=Java,
  linerange={12-15},
  firstnumber=12,
  caption={Include permission in theAndroidManifest.xml \cite{samsungZirconia}},
  label={codeSnippet:zirconiaPermission}
]{data/permission.xml}
The second addition is the implementation of the \textit{LicenseCheckListener}.
It contains the two results, either valid or invalid license verification result.
While \textit{licenseCheckedAsValid()} contains the code for success, \textit{licenseCheckedAsInvalid()} is used when the license cannot be validated.
\textit.
\lstinputlisting[
  float=h,
  basicstyle=\footnotesize,
  breakatwhitespace=false,
  breaklines=true,
  captionpos=b,
  frame=single,
  numbers=left,
  language=Java,
  linerange={85-101},
  firstnumber=85,
  caption={Zirconia license check callback},
  label={codeSnippet:zirconiaCallback}
]{data/samsung.java}
The third addition is initialization of the license check.
Zirconia handles all everything in its own.
The developer just has to set the listener for the result and start the check by calling the \textit{checkLicense()} method.
\newline
\lstinputlisting[
  float=h,
  basicstyle=\footnotesize,
  breakatwhitespace=false,
  breaklines=true,
  captionpos=b,
  frame=single,
  numbers=left,
  language=Java,
  linerange={56-61},
  firstnumber=56,
  caption={Setting up the Zirconia license check call},
  label={codeSnippet:zirconiaSetup}
]{data/samsung.java}
Zirconia always follows the same internal pattern when the license check is executed.
First, it is queried for a stored license.
If a stored license exists and it is valid, the check passes and no internet connection is required.
Otherwise Zirconia sends information of the device and the application to the server.
The server evaluates whether the user is authorized to use the application and replies accordingly.
The response is unique for each device and application combination and thus cannot be used on another device.
In case the access is granted, Zirconia stores the license on the device.
The next time the license check is initiated, the same flow is done.
