\subsection{Comparison of Code using Diff} \label{subsection:forensics-tools-diff}
The amount of code generated for the abstraction levels cannot be analysed without extra tools.
The best way to recover the changes is to use diff.
Diff is a standard command line tool which is used to compare two files in order to receive the differences between them.
\newline
The changes \gls{luckypatcherg} applied have to be identified.
This is achieved by comparing the different code abstractions of the original \gls{apk} with the cracked application.
Diff is used in a script to generate the result for different applications and abstraction levels at once (see code snippet~\ref{codeSnippet:diffScript}).
Doing this automatically and using diff saves a lot of time.
\newline
\lstinputlisting[
  float=h,
  breaklines=true,
  captionpos=b,
  frame=single,
  numbers=left,
  language=bash,
  linerange={1-9},
  firstnumber=1,
  caption={Script to compare the original and manipulated \gls{apk} to see the modifications in the different presentations},
  label={codeSnippet:diffScript}
]{data/diffScript.sh}
The result does not only contain the change as original and new code, but also the location where the change happened.
The example diff of a dex file is presented is code snippet~\ref{codeSnippet:n1DiffDex}.
\newline
\lstinputlisting[
  style=diff,
  breakatwhitespace=false,
  breaklines=true,
  captionpos=b,
  frame=single,
  linerange={1-3},
  caption={Diff on Dex level for N1 pattern},
  label={codeSnippet:n1DiffDex}
]{data/n1.diff}
