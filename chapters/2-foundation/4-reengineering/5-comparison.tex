\subsection{Detect Code Manipulations} \label{subsection:forensics-tools-diff}
The \textit{classes.dex} and its abstractions contain a lot of code.
The best way to recover the changes is to use diff.
Diff is a standard command line tool which is used to compare two files and in order to recieve the difference between them.
It is applied to the different code abstraction of the original \gls{apk} and the attacked \gls{apk}.
\newline
\lstinputlisting[
  float=h,
  breaklines=true,
  captionpos=b,
  frame=single,
  numbers=left,
  language=bash,
  linerange={1-9},
  firstnumber=1,
  caption={Script to compare the original and manipulated \gls{apk} to see the modifications in the different presentations},
  label={codeSnippet:diffScript}
]{data/diffScript.sh}
It returns the location as well as the orignal and changed code.
The automatic discovery and listing of the changes saves a lot of time.
The example diff of a dex file is presented is code snippet~\ref{codeSnippet:n1DiffDex}.
\newline
\lstinputlisting[
  style=diff,
  breakatwhitespace=false,
  breaklines=true,
  captionpos=b,
  frame=single,
  linerange={1-3},
  caption={Diff on Dex level for N1 pattern},
  label={codeSnippet:n1DiffDex}
]{data/n1.diff}
