\section{Code Analysis} \label{section:luckypatcher-analysis}
The code analysis was done using Lucky Patcher in version 6.0.4 using the two decompilers DAD and JADX, which are described in subsection~\ref{subsection:forensics-tools-java}.
The reverse engineered code was inspected using a text editor like Atom \cite{atom}.
Before analysing the code, a look is taken at the structure.
\newline
The folder containing the decompiled application is loaded into the editor.
On the first look, different folders can be spotted in the applications tructure.
These folders contain the resources, different libraries and the actual \gls{luckypatcherg} code.
The folders containing code can be divided into four categories.
\begin{enumerate}
\item Android Support Library v4 with many of it's modules -
It is used to provide downward compatibility of Android related functions.
\item Lucky Patcher code -
They are located in two places.
Utility functions, like \textit{copy file} and \textit{rename}, are stored in the package com.chelpus utility functions
The application code itself can be found in the package
\textit{com.android.vending.billing.InApp- BillingService.LACK}.
It contains the activities and functions which are used for cracking applications.
\item The third category are support libraries required by Lucky Patcher to apply it's patches -
This includes libraries, e.g. axml \cite{axml} for serializing the AndroidManifest.xml from Android binary into an ASCII formatted, human readable xml and zip4j \cite{zip4j}, a Java library to handle ZIP files.
\item The fourth is a modified billing and license library -
It is applied in combination with a proxy to redirect inapp billing and licensing calls.
\end{enumerate}
In the resources folder, different predefined custom patches can be found.
\newline
\newline
\gls{luckypatcherg} tries to hide the way it works.
The developer uses heavy obfuscation and a flat code structure to make reverse engineering time consuming.
It is applied effective since \gls{luckypatcherg} is dealing with applications which use this to fight off attacks.
For this reason a blackbox analysis is choosen.
The different modes are applied on a variety of applications and the outcome is analysed in order to identify how \gls{luckypatcherg} works.
