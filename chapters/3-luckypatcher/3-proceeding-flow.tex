\section{Analysis of Patched Applications} \label{section:luckypatcher-operation}
\begin{itemize}
  \item code analysis nto successful, analyse modified applications
  \item find attack points by using different apps and patching modes
  \item analysis with tools explained before
  \item use apks not odex since odex may be optimized
  \item goal is to see changes on different levels, dex, smali, java with diff
  \item start with reference application to ahve full control, lvl, samsung, amazon
  \item also on different apps to have a variety
  \item test outcome on different devices and evaluate
  \item test
  \item test
  \item test
  \item test
\end{itemize}
As described before, \gls{luckypatcherg} offers different modes to patch applications.
Each mode uses a set of patterns which each change a piece of binary.
These patterns are shown in figure~\ref{fig:luckyScreen} on the right.
A pattern is a set of predefined sequences of bytecode in which a certain values are modified.
In order to discover applied patterns and to evaluate each mode, each mode is applied on each application.
\newline
These are the different modes and what Lucky Patcher describes them as.
\begin{itemize}
\item The Auto Mode - "The minimal number of patches. Suitable for most applications with simple protection".
\item Auto Mode (Inversed) "There are a few differences from the ”Auto mode”. It may help you, if ”Auto mode” was unsuccessful."
\item Other Patches (Extreme Mode!) - "Additional patches (may cause instability). Apply only if the other patterns were unsuccessful. Requires internet. Try to use together with ”Auto mode” or ”Auto mode (Inversed)”."
\item Auto Mode (Amazon Market) - "Removes License Verification for applications from Amazon Market"
\item Auto Mode (SamsungApps) - "Removes License Verification for Apps from SamsungApps" (Note: SamsungApps is called GalaxyApps, see subsection~\ref{section:license-samsung})
\end{itemize}
