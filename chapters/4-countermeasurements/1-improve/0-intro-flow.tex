The first action is to fortify the spots identified in section~\ref{section:luckypatcher-patterns} as being attacked by \gls{luckypatcherg}.
The goal is to prevent the automatic application of patching patterns and stop execution in case tampering was detected.
Since additional checks can be voided by analysing the code manually and adding them to the patching procedure, obfuscation is introduced as a tool to make reverse engineering more complex.
