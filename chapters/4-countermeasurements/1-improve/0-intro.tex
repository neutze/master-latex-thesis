The first suggestion is to fortify the spots attacked by \gls{luckypatcherg} and identified in section~\ref{section:luckypatcher-patterns}.
The goal is to prevent the application of the bytecode pattern and thus the success of automatic patching.
This is done by changing the bytecode of the application, i.e. by modifying the \gls{lvl} implementation.
Additional checks are introduced and stop execution in case tampering was detected.
Since it is possible to alter code manually after analysing the it, obfuscation is introduced as a tool to make reverse engineering more time consuming.
