\subsubsection{Modify the Library} \label{subsection:counter-modifications-library}
STRUKTUR ÄNDERN
CODE ISOMORPH ÄNDERN. e.g. switch zu if
figure~\ref{fig:verificationNow} element (1)

most developers include the LVL unchanged without modifying it, attacks can always be done the same wayso this is very easy for luckypatcher
for own security and the security of other developers try to create a unique implementation,
ideal would be that no patterns can be applied to you and that in case the implementation gets racked no pattern of yours can be applied to others

this is a list of ideas of how to modify versus each pattern,

pattern 1,7 attacks the switch, the idea is to replace the switch with an if statemant or shuffle the cases
move it to a function and implement it somewhere as well thus the code might no longer be together with the rest of the class and the attacker has to specific search for it

pattern 2,4,5 skips using the outcome of a function and setting it always to true, this is a bit harder since it is already in a function, fitting bulk code around can make it harder to detect, especially for patterns, also checking inside again can help with detecting whether the fucntion is tampered, in case it is tampered the app can be killed or elements could not be loaded

pattern 3 modifies the return values on initialization, idea to alter the return value, changing it to e.g. int, thus all the ifs have to be modified to fit the needs


general ideas
move the lvl into own application folder
replace functions with inline code when possible
test again inside the function and e.g. kill the app on tampering
best would be to really make it different, e.g. recreate classes


creativity helps, does not protect at all but when applied correct automode does not work and patterns have to be registered

%eval
this tries to directly encounter luckypatcher by fighting the way the patterns work

it is easy to apply and there are no limits of variants how it can be modified
luckypatcher has to look at each app individual and in the worst case create custom patches
but as it seams developers do not like to modify as seen in analysis since all apps are somehow patchable
