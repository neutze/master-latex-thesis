When circumventing the \gls{lvl} the code has to be modified.
There are different indicators for actual and likely modifications on the application.
Indicators for actual modifications are forced debuggability, installation from a source different than the store or the changed signature of the application.
When \textit{root} is available or \gls{luckypatcherg} is installed on a device, these are indicators for likely modification.
\newline
These indicators can be tested and the program can be interrupted when in doubt.
For detected actual modifications no reason should be given for the killing of the application, as any reason given can be used for reengineering purposes.
Indicators of an unsafe environment should trigger interruption and a dialog, explaining the risk.
Of course interruption of the application results in a bad user experience.
\newline
These tests for these indicators are implemented as seen in figure~\ref{fig:verificationNowAdditional}.
\begin{figure}[h]
    \centering
    \includegraphics[width=0.8\textwidth]{data/verificationNowAdditional.png}
    \caption{Introduction of additional tests to check environment and integrity of the application}
    \label{fig:verificationNowAdditional}
\end{figure}
Since all these tampering countermeasures have the binary decision making similar to the license checks process.
They can be circumvented by turning them into the unary scenario by a reengineering effort.
/newline
On reengineering, the code has to be analysed in order to find, understand and patch them.
The countermeasures should be spread inside the application to unexpectedly crash the application.
The attacker not only has to invest time to figure out why the application crashes randomly, but also to find these checks.
Obfuscation, which should be applied as a standard, makes reengineering more difficult.
\newline
Any of this does not prevent \gls{luckypatcherg} from working, but it offers an additional layer of security.
