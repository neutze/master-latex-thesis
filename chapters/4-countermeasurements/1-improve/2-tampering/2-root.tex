\subsubsection{Root} \label{subsection:counter-improve-tampering-root}
\textit{Root} can be used to alter applications or extract protected data.
The developer can check whether \textit{root} is available on the device and eventually exclude users which use it.
The developer needs to communicate the users the reasons for this strict policy since there are a lot of users who use root for other reasons than cracking applications.
\newline
Google has introduced a similar \gls{api}, called SafetyNet \cite{safetynetPay}, which is said to be used in security critical applications like Android Pay \cite{safetynetGoogle} \cite{safetynetPayx}.
\newline
\lstinputlisting[
  float=h,
  basicstyle=\footnotesize,
  breakatwhitespace=false,
  breaklines=true,
  captionpos=b,
  frame=single,
  numbers=left,
  language=Java,
  linerange={16-37},
  firstnumber=16,
  caption={Example code for checking for root},
  label={codeSnippet:tamperingRoot}
]{data/4-countermeasurements/1-tampering/2-root.java}
Since \textit{root} is achieved by the \textit{su} file in the filesystem, the application can search for its existance in the common locations.
In case the search is successful, the execution of the application can be terminated.
