\subsubsection{Signature} \label{subsection:counter-improve-tampering-signature}
Application code is signed to connect a developer to an application and enable him to provide updates for the application.
Since the checksum and signature have to be rewritten when cracking the application, this can be used to detect attacks.
This is similar to Google Maps inside an application.
The Maps library sends the signature and the API key to the server which. similar to the license verification, decides wether the application is allowed to show the map.
\newline
\lstinputlisting[
  float=h,
  basicstyle=\footnotesize,
  breaklines=true,
  captionpos=b,
  frame=single,
  numbers=left,
  language=Java,
  linerange={51-74},
  firstnumber=51,
  caption={Example code for checking the signature of the application},
  label={codeSnippet:tamperingSignature}
]{data/4-countermeasurements/1-tampering/5-signature.java}
The code for the signature check can be seen in code snippet~\ref{codeSnippet:tamperingSignature}.
In order to check the application's signature, the original signature has to be provided (see line 53).
The application's signature fetched from the package information (line 57ff).
In case the signature cannot be retrieved or the signature do not match, the check terminates the application.
