\subsubsection{Encryption} \label{subsection:counter-replace-encryption-content}
Encryption can be applied on different levels inside the application.
The thesis introduces three different approaches on encryption.

\subsubsection{Encryption - Resources} \label{subsection:counter-replace-encryption-content-resource}
The first approach is to apply encryption on the application's static resources.
This can include the application's hard coded strings or image assets.
Whenever a resource is used, it has to be decrypted first.
\newline
If application critical strings are encrypted, like server addresses are encrypted, the application is unable to work.
If the strings are output strings or pictures, the application will work, but the user will not understand the output, because they are still encrypted and have no meaning to the user.
\newline
Figure~\ref{fig:encryptionResource} shows the abstract implementation of resource decryption.
\begin{figure}[h]
    \centering
    \includegraphics[width=0.8\textwidth]{data/encryptionResource.png}
    \caption{Encrypted resources have to be decrypted before they are used or displayed}
    \label{fig:encryptionResource}
\end{figure}
There is a possibility to steal the resources from authorized applications and insert them in a recreated \gls{apk}.
This is a significant reengineering effort for a single application.

\subsubsection{Encryption - Action Obfuscator} \label{subsubsectionection:counter-replace-encryption-content-obfuscator}
The second approach is to use encryption as a form of obfuscation.
The idea is to have a single method to delegate all other method calls according to an encrypted parameter.
\newline
When an attacker does a static analysis of the code, the link between the initial call and target method are not apparent.
It forces the attacker to use a dynamic analysis method instead.
The fortified mechanism even more when encrypted arguments are passed.
\newline
An abstract presentation of the mechanism can be seen in figure~\ref{fig:encryptionAction}.
\begin{figure}[h]
    \centering
    \includegraphics[width=0.8\textwidth]{data/encryptionAction.png}
    \caption{Encrypted actions to obfuscate dependencies}
    \label{fig:encryptionAction}
\end{figure}

\subsubsection{Encryption - Communication} \label{section:counter-replace-encryption-content-communication}
The third approach is to use encryption on the server response as seen in figure~\ref{fig:encryptionComm}.
This additional security feature is applied in combination with a content server as described in subsection~\ref{section:counter-replace-server}.
\newline
\begin{figure}[h]
    \centering
    \includegraphics[width=0.8\textwidth]{data/encryptionComm.png}
    \caption{Encrypted communication with a server}
    \label{fig:encryptionComm}
\end{figure}
