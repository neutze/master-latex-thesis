\subsubsection{Proguard}\label{subsubsection:counter-reengineering-optobf-proguard}
\url{https://youtu.be/6vFcEJ2jgOw?t=419}\newline

\url{http://developer.android.com/tools/help/proguard.html}\newline
optimizes, shrinks, (barely) obfuscates --> free, reduces size, faster\newline
gutes bild \url{https://youtu.be/TNnccRimhsI?t=1360}\newline
removes unnecessary/unused code\newline
merges identical code blocks\newline
performs optimiztations\newline
removes debug information\newline
renames objects\newline
restructures code\newline
removes linenumbers --> stacktrace annoying\newline
\url{https://youtu.be/6vFcEJ2jgOw?t=470}\newline
-->hacker factor 0\newline
does not really help\newline
googles commentar \url{http://android-developers.blogspot.de/2010/09/proguard-android-and-licensing-server.html}

\url{https://net.cs.uni-bonn.de/fileadmin/user_upload/plohmann/2012-Schulz-Code_Protection_in_Android.pdf}
ProGuard [4] is an open source tool which is also integrated in the Android SDK [5]. It can
be easily used within the development process. ProGuard is basically a Java obfuscator
but can also be used for Android applications because they are usually written in Java.
The feature set includes identifier obfuscation for packages, classes, methods, and fields.
Besides these protection mechanisms it can also identify and highlight dead code so it can
be removed in a second, manual step. Unused classes can be removed automatically by
ProGuard.
Without proper naming of classes and methods it is much harder to reverse engineer
an application, because in most cases the identifier enables an analyst to directly guess
the purpose of the particular part. The program code itself will not be changed heavily,
so the obfuscation by this tool is very limited.\newline
E. Lafortune. Proguard. Visited: May, 2012. [Online]. Available: \url{http:
//proguard.sourceforge.net/}
