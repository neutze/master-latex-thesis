\subsection{Secure Elements}\label{subsection:counter-external-secure}
%START TEXT INPUT
This is my real text! Rest might be copied or not be checked!
%START TEXT INPUT

WAS IST ES?
WAS MACHT ES?
WIE IMPLEMENTIERT MAN?



ERWÄHNEN WO IM PROZESS ANGEWENDET\newline

new section trusted execution environment
trusttronic letzte conference
samsung knox
--see-gelten eher sicher

\url{https://usmile.at/sites/default/files/androidsecuritysymposium/presentations/Thomas_Holmes_AnInfestationOfDragons.pdf}


TEE!!!

was ist dann geschützt? content, servers, time constrained urls, obfuscation by using reflection combined with SE -see- makes slow but no static analysis\newline

very very slow, e.g 10kHz so no big calculations possible\newline
250bytes, 200ms \newline

\url{http://amies-2014.international-symposium.org/proceedings_2014/Kannengiesser_Baumgarten_Song_AmiEs_2014_Paper.pdf}\newline

DAP Verification .... normalerweise muss jede Applet, die auf so ein Secure Element/Smartcard etc. kommt mit ner Signatur unterschrieben sein ...
%\url{http://www.win.tue.nl/pinpasjc/docs/Card%20Spec%20v2.1.1%20v0303.pdf}


Waehrend ich Exploits finden konnte, die Dir erw. Zugriff geben, wenn du Applets installieren kannst, u.a.
%\url{https://www.cs.ru.nl/E.Poll/papers/cardis08.pdf}
%\url{http://www.uclouvain.be/crypto/wissec2009/static/13.pdf}​
