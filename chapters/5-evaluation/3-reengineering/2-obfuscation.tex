\subsection{Obfuscation}
\label{subsection:evaluation-reengineering-optobf}

in theory a good addition to the security of the application, but against luckypatcher directly since it works on java level in order to disguise the way the code works
it enforces increased initial effort an attacker has to spend in order to understand the code and thus reduces the likelihood of attackers to being motivated to crack the application
in practice obfuscators are limited due to:
reliance on android framework apis (remain unobfuscated)
applications can be debugged
jdwp and application debuggability at the java lvl can reveal information about the software
popular enough obfuscators (dexguard) have deobfuscators
professional tools cost a lot of money, you have to be commercial or have a real good idea in order to be worthwhile
