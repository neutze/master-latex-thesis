%http://tex.stackexchange.com/questions/8946/how-to-combine-acronym-and-glossary
%use \gls{asd}
\newglossaryentry{adbg}{
    name={ADB},
    description={The Android Debug Bridge is a command-line application providing different debugging tools}
}

\newglossaryentry{apkg}{
    name={APK},
    description={An Android Application Package is the file format used for distributing and installing applications on the Android operating system. It contains the applications assets, code (.dex file), manifest and resources}
}

\newglossaryentry{classg}{
    name={.class},
    description={Java Byte Code produced by the Java compiler from a .java file}
}

\newglossaryentry{dexg}{
    name={.dex},
    description={Dalvik Byte Code file, translated from the Java bytecode. Dalvik Executables are designed to run on system with memory or processor constraints. For example, the .dex file of the Phone application is inside the system/app/Phone.apk}
}

\newglossaryentry{odexg}{
    name={.odex},
    description={Optimized Dalvik Byte Code file are Dalvik Executables optimized for the current device the application is running on. For example, the .odex file of the Phone application is system/app/Phone.odex}
}

\newglossaryentry{assemblerg}{
    name={assembler},
    description={Ein Assembler (auch Assemblierer[1]) ist ein Computerprogramm, das Assemblersprache in Maschinensprache übersetzt, beispielsweise den Assemblersprachentext „CLI“ in den Maschinensprachentext „11111010“.}
}

\newglossaryentry{disassemblerg}{
    name={disassembler},
    description={Ein Disassembler ist ein Computerprogramm, das die binär kodierte Maschinensprache eines ausführbaren Programmes in eine für Menschen lesbarere Assemblersprache umwandelt. Seine Funktionalität ist der eines Assemblers entgegengesetzt.}
}
